\chapter{Statistical Inference}
In real life, we work with data that are affected by randomness, and we need to extract information and draw conclusions from the data. The randomness might come from a variety of sources. 

\textit{Statistical inference} is a collection of methods that deal with drawing conclusions from data that are prone to random variation.

\textbf{Frequentist (classical) Inference}: In this approach, the unknown quantity $\theta$ is assumed to be a fixed quantity. That is, $\theta$ is a deterministic (non-random) quantity that is to be estimated by the observed data. For example, in the polling problem stated above we might consider $\theta$ as the percentage of people who will vote for a certain candidate, call him/her Candidate A. After asking $n$ randomly chosen voters, we might estimate $\theta$ by
\begin{align*}
	\theta = \frac{Y}{n},
\end{align*}
where $Y$ is the number of people who support for candidate $A$.

\textbf{Bayesian Inference}: In the Bayesian approach the unknown quantity $\theta$ is assumed to be a random variable, and we assume that we have some initial guess about the distribution of $\theta$. After observing the data, we update the distribution of $\theta$ using Bayes' Rule.
